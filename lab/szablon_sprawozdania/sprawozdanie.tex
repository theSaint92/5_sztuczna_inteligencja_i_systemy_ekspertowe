% vim:encoding=utf8 ft=tex sts=2 sw=2 et:

\documentclass{classrep}
\usepackage[utf8]{inputenc}

\studycycle{Informatyka, studia dzienne, mgr II st.}
\coursesemester{III}

%\coursename{Angelologia teoretyczna i stosowana}
\coursename{Programowanie współbieżne}
\courseyear{2009/2010}

\courseteacher{prof. Bączyński}
\coursegroup{wtorek, 14:15}

\author{
  \studentinfo{Julian Brzechwa}{540282} \and
  \studentinfo{Jan Mickiewicz}{823941} \and
  \studentinfo{Adam Tuwim}{561004}
}

\title{Zadanie 1: Wybrane aspekty niektórych zagadnień}
\svnurl{http://serce.ics.p.lodz.pl/svn/labs/atis/b_wt1415/grupa/zad1@134}

\begin{document}
\maketitle

\section{\LaTeX?}
Jeśli skład tego sprawozdania jest jednym z pierwszych kontaktów z
\LaTeX\dywiz em, warto zapoznać się z ,,Nie za krótkim wprowadzeniem do
systemu \LaTeX2e''\cite{l2short}.

\section{Instrukcja obsługi szablonu}
Szablon należy czytelnie wypełnić treścia. Proszę pisać zwięźle i na temat.

Klasa sprawozdania wymaga następujących pakietów:
\begin{itemize}
  \item zestaw klas autorstwa Marcina Wolińskiego \ppauza używana jest klasa
    \emph{mwart} (nazwa całego pakietu: \emph{mwcls}),
  \item pakiet \emph{polski},
  \item pakiet \emph{url}.
\end{itemize}

W preambule należy wykorzystać następujące niestandardowe polecenia, aby
dostarczyć informacje wymagane do wygenerowania sprawozdania:
\begin{itemize}
  \item \verb+\studycycle{...}+ \ppauza tryb i rodzaj studiów,
  \item \verb+\coursesemester{...}+ \ppauza semestr studiów,
  \item \verb+\coursename{...}+ \ppauza nazwa przedmiotu,
  \item \verb+\courseyear{...}+ \ppauza bieżący rok akademicki,
  \item \verb+\courseteacher{...}+ \ppauza prowadzący laboratoria (proszę
    wpisywać osobę, której oddaje się sprawozdanie a nie kierownika
    przedmiotu!),
  \item \verb+\coursegroup{...}+ \ppauza identyfikator grupy ćwiczeniowej,
  \item \verb+\studentinfo[e-mail]{imię i nazwisko}{nr albumu}+ \ppauza
    polecenie wykorzystywane \emph{tylko} w obrębie polecenia \verb+\author+
    służy do określenia imienia i nazwiska każdego z autorów jak również
    numeru albumu i opcjonalnie adresu e\dywiz mail; jeśli adres nie jest
    podawany, należy pominąć argument opcjonalny.
\end{itemize}
Można również podać URL do repozytorium zawierającego kod programu. Należy w
tym celu wykorzytać polecenie \verb+\svnurl{...}+. Jeśli istotne jest
zaznaczenie numeru rewizji, jej numer należy podać oddzielając go od reszty
URL-a znakiem \verb+@+. Jeśli repozytorium znajduje się na serwerach Instytutu
można podać tylko skróconą nazwę, czyli np. \url{serce} zamiast
\url{serce.ics.p.lodz.pl}.

\paragraph{Uwaga:} po wydrukowaniu proszę nic nie wpisywać w pola
,,Data oddania'' i ,,Ocena''.

\section{Przydatne pakiety}
Przy składaniu sprawozdania mogą przydać się pakiety wchodzące w
skład dystrybucji \TeX\dywiz a. W szczególności warte uwagi są:
\begin{itemize}
  \item \emph{amsmath}/\emph{amsfonts} i inne pakiety \emph{ams*} \ppauza
    zestaw dodatkowych poleceń i~symboli do wykorzystania przy składzie
    wzorów,
  \item \emph{algorithm}, \emph{algorithmic} \ppauza skład algorytmów,
  \item \emph{dcolumn} \ppauza ułatwia skład tabel, w których liczby
    wyrównywane są do separatora dziesiętnego,
  \item \emph{enumerate} \ppauza bardziej wyrafinowany skład wyliczeń,
  \item \emph{graphicx} \ppauza polecenia do włączania grafiki,
  \item \emph{icomma} \ppauza pozwala na wykorzystanie przecinka jako
    separatora dziesiętnego w trybie matematycznym,
  \item \emph{listings} \ppauza skład kodów źródłowych programów,
  \item \emph{url} \ppauza skład URL-i.
\end{itemize}

\begin{thebibliography}{0}
  \bibitem{l2short} T. Oetiker, H. Partl, I. Hyna, E. Schlegl.
    \textsl{Nie za krótkie wprowadzenie do systemu \LaTeX2e}, 2007, dostępny
    online.
\end{thebibliography}
\end{document}
